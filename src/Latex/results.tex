% Results and Discussion chapter for thesis
% This file will be included via % Results and Discussion chapter for thesis
% This file will be included via % Results and Discussion chapter for thesis
% This file will be included via % Results and Discussion chapter for thesis
% This file will be included via \input{results} in main thesis.tex

\chapter{Results and Discussion}
\label{chap:results_discussion}

This thesis compiles the design, implementation and verification of a pipeline floating-point processor using SystemC with FPGA prototyping on Zynq platform.

\section{Research Contributions}
\label{sec:research_contributions}

\begin{itemize}
    \item Established a comprehensive design for a pipelined floating-point unit with proper pipeline stages and correct implementation of IEEE 754 operations.
    
    \item The research introduced SystemC modeling for floating point operations to provide early functional verification to hardware description.
    
    \item We demonstrated easy workflow using Intel Compiler for SystemC to translate high level SystemC models into Synthesizable Verilog constructs.
    
    \item We implemented a convenient and extensive verification methodology combining C-based testing, System Verilog simulation and waveform analysis to ensure minimal errors.
    
    \item Addressed hardware-software integration challenges, timing issues with consistent exception handling in all stages.
\end{itemize}

\begin{figure}[h]
\centering
\includegraphics[width=0.9\textwidth]{figures/FP_pipelined.png}
\caption{Implemented floating-point pipelined processor architecture overview}
\label{fig:fp_pipelined}
\end{figure}

These contributions take care of multiple obstacles in floating point hardware design, focusing on the methodological aspects discussed in development from modeling to verification to outputting complex hardware designs.

\section{Key Findings and Insights}
\label{sec:key_findings}

Our research has found several things regarding pipelined floating-point design and implementation:

\begin{itemize}
    \item The five-stage pipeline gives an effective balance between throughput and implementation ease. Separating the IEEE 754 operations across stages enables simpler handling of arithmetic calculations.
    
    \item The floating-point unit's preparation for implementation on FPGA platforms has been conveniently achieved through RTL implementation and synthesis preparation. It is a solid start, but more testing needs to be done on a variety of FPGA platforms.
    
    \item Early detection and processing of IEEE 754 special cases (zero, infinity, NaN) throughout pipeline significantly reduced errors.
\end{itemize}

\section{Limitations of Our Work}
\label{sec:limitations}

\begin{itemize}
    \item The implementation is done only on 32-bit single precision format. So, there is always scope to extend it to 16-bit and 64-bit formats as well based on applications.
    
    \item The current design has only four basic floating-point operations and can later be extended using rv32f instructions to form a proper RISC-V processor with square root, branch, jump instructions and integer arithmetic operations.
    
    \item The research gave more priority to functional correctness than performance optimization which can be improved in future.
    
    \item The verification approach relied primarily on simulation-based techniques leaving out formal verification methods, which could provide more stronger mathematical correctness.
    
    \item The implementation is specifically targeted for Xilinx Zynq board due to time constraints and more work needs to be done to make the prototype suitable for boards.
\end{itemize}

\section{Future Research Directions}
\label{sec:future_directions}

\begin{itemize}
    \item Extending precision support to 64-bit operations would broaden the applications of this floating-point processors to scientific applications.
    
    \item Implementing additional operations such as square root, logarithmic functions and trigonometric functions would make it more of an ALU unit for scientific and graphics applications.
    
    \item Detailed performance optimization, like critical path analysis, pipeline stage rebalancing, and throughput calculations would improve the implementation efficiency.
    
    \item Checking out power optimization techniques such as clock gating, operand isolation, and activity-aware resource allocation would help take steps towards energy efficiency which is also important these days.
    
    \item Exploring better ways to integrate between floating point hardware and software algorithms will give better system performance for targeted applications.
\end{itemize} in main thesis.tex

\chapter{Results and Discussion}
\label{chap:results_discussion}

This thesis compiles the design, implementation and verification of a pipeline floating-point processor using SystemC with FPGA prototyping on Zynq platform.

\section{Research Contributions}
\label{sec:research_contributions}

\begin{itemize}
    \item Established a comprehensive design for a pipelined floating-point unit with proper pipeline stages and correct implementation of IEEE 754 operations.
    
    \item The research introduced SystemC modeling for floating point operations to provide early functional verification to hardware description.
    
    \item We demonstrated easy workflow using Intel Compiler for SystemC to translate high level SystemC models into Synthesizable Verilog constructs.
    
    \item We implemented a convenient and extensive verification methodology combining C-based testing, System Verilog simulation and waveform analysis to ensure minimal errors.
    
    \item Addressed hardware-software integration challenges, timing issues with consistent exception handling in all stages.
\end{itemize}

\begin{figure}[h]
\centering
\includegraphics[width=0.9\textwidth]{figures/FP_pipelined.png}
\caption{Implemented floating-point pipelined processor architecture overview}
\label{fig:fp_pipelined}
\end{figure}

These contributions take care of multiple obstacles in floating point hardware design, focusing on the methodological aspects discussed in development from modeling to verification to outputting complex hardware designs.

\section{Key Findings and Insights}
\label{sec:key_findings}

Our research has found several things regarding pipelined floating-point design and implementation:

\begin{itemize}
    \item The five-stage pipeline gives an effective balance between throughput and implementation ease. Separating the IEEE 754 operations across stages enables simpler handling of arithmetic calculations.
    
    \item The floating-point unit's preparation for implementation on FPGA platforms has been conveniently achieved through RTL implementation and synthesis preparation. It is a solid start, but more testing needs to be done on a variety of FPGA platforms.
    
    \item Early detection and processing of IEEE 754 special cases (zero, infinity, NaN) throughout pipeline significantly reduced errors.
\end{itemize}

\section{Limitations of Our Work}
\label{sec:limitations}

\begin{itemize}
    \item The implementation is done only on 32-bit single precision format. So, there is always scope to extend it to 16-bit and 64-bit formats as well based on applications.
    
    \item The current design has only four basic floating-point operations and can later be extended using rv32f instructions to form a proper RISC-V processor with square root, branch, jump instructions and integer arithmetic operations.
    
    \item The research gave more priority to functional correctness than performance optimization which can be improved in future.
    
    \item The verification approach relied primarily on simulation-based techniques leaving out formal verification methods, which could provide more stronger mathematical correctness.
    
    \item The implementation is specifically targeted for Xilinx Zynq board due to time constraints and more work needs to be done to make the prototype suitable for boards.
\end{itemize}

\section{Future Research Directions}
\label{sec:future_directions}

\begin{itemize}
    \item Extending precision support to 64-bit operations would broaden the applications of this floating-point processors to scientific applications.
    
    \item Implementing additional operations such as square root, logarithmic functions and trigonometric functions would make it more of an ALU unit for scientific and graphics applications.
    
    \item Detailed performance optimization, like critical path analysis, pipeline stage rebalancing, and throughput calculations would improve the implementation efficiency.
    
    \item Checking out power optimization techniques such as clock gating, operand isolation, and activity-aware resource allocation would help take steps towards energy efficiency which is also important these days.
    
    \item Exploring better ways to integrate between floating point hardware and software algorithms will give better system performance for targeted applications.
\end{itemize} in main thesis.tex

\chapter{Results and Discussion}
\label{chap:results_discussion}

This thesis compiles the design, implementation and verification of a pipeline floating-point processor using SystemC with FPGA prototyping on Zynq platform.

\section{Research Contributions}
\label{sec:research_contributions}

\begin{itemize}
    \item Established a comprehensive design for a pipelined floating-point unit with proper pipeline stages and correct implementation of IEEE 754 operations.
    
    \item The research introduced SystemC modeling for floating point operations to provide early functional verification to hardware description.
    
    \item We demonstrated easy workflow using Intel Compiler for SystemC to translate high level SystemC models into Synthesizable Verilog constructs.
    
    \item We implemented a convenient and extensive verification methodology combining C-based testing, System Verilog simulation and waveform analysis to ensure minimal errors.
    
    \item Addressed hardware-software integration challenges, timing issues with consistent exception handling in all stages.
\end{itemize}

\begin{figure}[h]
\centering
\includegraphics[width=0.9\textwidth]{figures/FP_pipelined.png}
\caption{Implemented floating-point pipelined processor architecture overview}
\label{fig:fp_pipelined}
\end{figure}

These contributions take care of multiple obstacles in floating point hardware design, focusing on the methodological aspects discussed in development from modeling to verification to outputting complex hardware designs.

\section{Key Findings and Insights}
\label{sec:key_findings}

Our research has found several things regarding pipelined floating-point design and implementation:

\begin{itemize}
    \item The five-stage pipeline gives an effective balance between throughput and implementation ease. Separating the IEEE 754 operations across stages enables simpler handling of arithmetic calculations.
    
    \item The floating-point unit's preparation for implementation on FPGA platforms has been conveniently achieved through RTL implementation and synthesis preparation. It is a solid start, but more testing needs to be done on a variety of FPGA platforms.
    
    \item Early detection and processing of IEEE 754 special cases (zero, infinity, NaN) throughout pipeline significantly reduced errors.
\end{itemize}

\section{Limitations of Our Work}
\label{sec:limitations}

\begin{itemize}
    \item The implementation is done only on 32-bit single precision format. So, there is always scope to extend it to 16-bit and 64-bit formats as well based on applications.
    
    \item The current design has only four basic floating-point operations and can later be extended using rv32f instructions to form a proper RISC-V processor with square root, branch, jump instructions and integer arithmetic operations.
    
    \item The research gave more priority to functional correctness than performance optimization which can be improved in future.
    
    \item The verification approach relied primarily on simulation-based techniques leaving out formal verification methods, which could provide more stronger mathematical correctness.
    
    \item The implementation is specifically targeted for Xilinx Zynq board due to time constraints and more work needs to be done to make the prototype suitable for boards.
\end{itemize}

\section{Future Research Directions}
\label{sec:future_directions}

\begin{itemize}
    \item Extending precision support to 64-bit operations would broaden the applications of this floating-point processors to scientific applications.
    
    \item Implementing additional operations such as square root, logarithmic functions and trigonometric functions would make it more of an ALU unit for scientific and graphics applications.
    
    \item Detailed performance optimization, like critical path analysis, pipeline stage rebalancing, and throughput calculations would improve the implementation efficiency.
    
    \item Checking out power optimization techniques such as clock gating, operand isolation, and activity-aware resource allocation would help take steps towards energy efficiency which is also important these days.
    
    \item Exploring better ways to integrate between floating point hardware and software algorithms will give better system performance for targeted applications.
\end{itemize} in main thesis.tex

\chapter{Results and Discussion}
\label{chap:results_discussion}

This thesis compiles the design, implementation and verification of a pipeline floating-point processor using SystemC with FPGA prototyping on Zynq platform.

\section{Research Contributions}
\label{sec:research_contributions}

\begin{itemize}
    \item Established a comprehensive design for a pipelined floating-point unit with proper pipeline stages and correct implementation of IEEE 754 operations.
    
    \item The research introduced SystemC modeling for floating point operations to provide early functional verification to hardware description.
    
    \item We demonstrated easy workflow using Intel Compiler for SystemC to translate high level SystemC models into Synthesizable Verilog constructs.
    
    \item We implemented a convenient and extensive verification methodology combining C-based testing, System Verilog simulation and waveform analysis to ensure minimal errors.
    
    \item Addressed hardware-software integration challenges, timing issues with consistent exception handling in all stages.
\end{itemize}

\begin{figure}[h]
\centering
\includegraphics[width=0.9\textwidth]{figures/FP_pipelined.png}
\caption{Implemented floating-point pipelined processor architecture overview}
\label{fig:fp_pipelined}
\end{figure}

These contributions take care of multiple obstacles in floating point hardware design, focusing on the methodological aspects discussed in development from modeling to verification to outputting complex hardware designs.

\section{Key Findings and Insights}
\label{sec:key_findings}

Our research has found several things regarding pipelined floating-point design and implementation:

\begin{itemize}
    \item The five-stage pipeline gives an effective balance between throughput and implementation ease. Separating the IEEE 754 operations across stages enables simpler handling of arithmetic calculations.
    
    \item The floating-point unit's preparation for implementation on FPGA platforms has been conveniently achieved through RTL implementation and synthesis preparation. It is a solid start, but more testing needs to be done on a variety of FPGA platforms.
    
    \item Early detection and processing of IEEE 754 special cases (zero, infinity, NaN) throughout pipeline significantly reduced errors.
\end{itemize}

\section{Limitations of Our Work}
\label{sec:limitations}

\begin{itemize}
    \item The implementation is done only on 32-bit single precision format. So, there is always scope to extend it to 16-bit and 64-bit formats as well based on applications.
    
    \item The current design has only four basic floating-point operations and can later be extended using rv32f instructions to form a proper RISC-V processor with square root, branch, jump instructions and integer arithmetic operations.
    
    \item The research gave more priority to functional correctness than performance optimization which can be improved in future.
    
    \item The verification approach relied primarily on simulation-based techniques leaving out formal verification methods, which could provide more stronger mathematical correctness.
    
    \item The implementation is specifically targeted for Xilinx Zynq board due to time constraints and more work needs to be done to make the prototype suitable for boards.
\end{itemize}

\section{Future Research Directions}
\label{sec:future_directions}

\begin{itemize}
    \item Extending precision support to 64-bit operations would broaden the applications of this floating-point processors to scientific applications.
    
    \item Implementing additional operations such as square root, logarithmic functions and trigonometric functions would make it more of an ALU unit for scientific and graphics applications.
    
    \item Detailed performance optimization, like critical path analysis, pipeline stage rebalancing, and throughput calculations would improve the implementation efficiency.
    
    \item Checking out power optimization techniques such as clock gating, operand isolation, and activity-aware resource allocation would help take steps towards energy efficiency which is also important these days.
    
    \item Exploring better ways to integrate between floating point hardware and software algorithms will give better system performance for targeted applications.
\end{itemize}